%% Please fill in your name and collaboration statement here.
\newcommand{\studentName}{**FILL IN YOUR NAME HERE**}
\newcommand{\collaborationStatement}{**FILL IN YOUR COLLABORATION STATEMENT HERE \\ (See the syllabus for information)**}


%%%%%%%%%%%%%%%%%%%%%%%%%%%%%%%%%%%%%%%%%%%%%%%
\documentclass[solution, letterpaper]{cscie121}
\usepackage{enumerate}
\usepackage{tikz}
\usepackage{pgf}
\usepackage{tikz}
\usetikzlibrary{arrows,automata}
\usepackage{hyperref}
\usetikzlibrary{automata,positioning}
\begin{document}
\header{4}{Due October 23, 2015}


\problem{3+3+3} {1/2 page}
For each of the following languages, state whether the language is context-free or not. If context-free, give a context-free grammar that generates the language (or construct a pushdown automaton that recognizes the language). If not context-free, write a short proof to justify.

\subproblem $\{a^{n} : n \text{ is a prime number}\}$. 
\subproblem $\{a^nb^m : n,m\in \mathbb{N},  n \neq m\; \textrm{and}\; n \neq 2m\}$.
\subproblem
$\{a^{n}b^{n}c^{n} : n\in\mathbb{N}\} \cup \{(ab)^n(ca)^n(cb)^n : n\in\mathbb{N}\}$. 

\begin{solution}
  % Write your answer here.
\end{solution}


\problem{6}{1/2 page}
Draw the state diagram of a PDA that recognizes the language $\{w :$ the number
of $a$'s in $w$ is greater than two times the number of $b$'s in $w\}$. Use
Sipser's notation to label the transitions. (That is, use labels in the form
$x,y\to z$ to mean that when you read an $x$ from your string and pop a $y$
from the stack, follow the transition and push a $z$ onto the stack.) Explain
in a few sentences why your construction is correct (no need to prove
formally).

\begin{solution}
  % Write your answer here.
\end{solution}


\problem{6}{1/2 page}
A pushdown automaton (PDA) is made by equipping an NFA with a stack. In this
problem, we consider an NFA equipped with a queue -- a Queue Automaton (QA).
The queue supports the following two operations:
\begin{enumerate}
  \item {\em Pop.} Read the leftmost symbol of the queue and remove it from
    the queue.
  \item {\em Push.} Write a symbol to the rightmost end of the queue.
\end{enumerate}

In a Queue Automaton, a transition $(q, \sigma, \gamma)\rightarrow (q',
\gamma')$ means ``if the machine is in state $q$, after reading $\sigma$ from
the input string and popping $\gamma$ from the queue, push
$\gamma'$ onto the queue and transition to state $q'$''. Note that
$\varepsilon$-transitions are still allowed. For example, $(q, \varepsilon,
\gamma)\rightarrow (q', \gamma')$ doesn't read anything from the input string.
You can also use an $\varepsilon$ for $\gamma$ to indicate that nothing should
be popped from the queue, and an $\varepsilon$ for $\gamma'$ to indicate that
nothing should be pushed onto the queue. Initially the queue is empty.

We saw in class that $L=\{a^{2^n} : n \in \Nat \}$ is not Context-Free, so
there isn't a PDA that recognizes it. Construct a Queue Automaton that {\em
does} recognize $L$ (give the 6-tuple for it) and provide a short explanation
for why it works.

\begin{solution}
  % Write your answer here.
\end{solution}


\problem{4+6+(2)}{1 page}
Recall that allowing nondeterminism did not add any power to finite 
automata---any language that could be accepted by an NFA could also be 
accepted by a DFA. In this problem, you will show that this is \emph{not 
the case} for PDAs by defining Deterministic PDAs (DPDAs). \\ 

DPDAs are similar to PDAs with a few subtle differences: 
\begin{itemize}
\item Most notably, they are deterministic in their transitions and
  stack operations.  Note that this means that they don't allow
  transitions on $\varepsilon$.  In one step, a DPDA reads exactly one
  input symbol, reads and pops exactly one symbol from the top of the
  stack, and pushes a {\em string} onto the stack.
\item Instead of starting with an empty stack, they start with a
  special symbol, $\$$, as the lone symbol on the stack. This way a
  DPDA can always know when it has reached the bottom of its
  stack. The $\$$ symbol can never be erased; that is, any transition
  reading $\$$ \emph{must} push it back on the stack.
\end{itemize}

\subproblem  Formalize the above by definining a DPDA as a 7-tuple $(Q,
\Sigma, \Gamma, \$, \delta, q_0, F)$, filling in the requirements of
each component.  You can refer to the definitions of a regular PDA for
the components that don't change.
\subproblem Now you will show that DPDAs are more powerful than finite automata. Consider the language $\textsc{Majority} =\{w: w \text{ contains more
$a$'s than  $b$'s}\}$. Construct a DPDA that recognizes $\textsc{Majority}$. Prove that no finite automata recognizes $\textsc{Majority}$.
\subproblem (CHALLENGE!!! Optional, worth 2 extra points.) Show that DPDAs are less powerful than PDAs.

\begin{solution}
  % Write your answer here.
\end{solution}

\problem{8}{3/4 page}
If $L$ is a language, then {\sc Permutation}$(L) = \{ x:
$ there exists a string $w \in L$ such that $|x|=|w|$ and the number
of occurrences of any letter in $w$ and $x$ are the same$\}$. Show
that if $L$ is regular, then {\sc Permutation}(L) is context free
when $\Sigma = \{a, b\}$. (Hint: Your proof must use the fact that there are only two symbols in the alphabet---the proof does not generalize to the case $\lvert \Sigma\rvert >2$.)

\begin{solution}
  % Write your answer here.
\end{solution}
\end{document}

