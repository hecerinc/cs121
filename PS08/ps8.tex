%% Please fill in your name and collaboration statement here.
\newcommand{\studentName}{Kevin Zhang}
\newcommand{\collaborationStatement}{I collaborated with Tomoya Hasegawa and Jessica Zhu, got help from no one else, and referred to no other resources.}


%%%%%%%%%%%%%%%%%%%%%%%%%%%%%%%%%%%%%%%%%%%%%%%
\documentclass[solution, letterpaper]{cs121}
\usepackage{enumerate}
\usepackage{tikz}
\usepackage{pgf}
\usepackage{tikz}
\usetikzlibrary{arrows,automata}
\usepackage{hyperref}
\usetikzlibrary{automata,positioning}
\begin{document}
\header{8}{Due November 17, 2015 11:59pm}


%%%%%%%%%%%%%%%%%%%%%%%%%%%%%%%%%%%%%%%%%%%%%%%
\PART{Zhengyu and Geoff}
%%%%%%%%%%%%%%%%%%%%%%%%%%%%%%%%%%%%%%%%%%%%%%%

\problem{2+2+2+2+2+2}{2/3 page} 
For each part (A) through (F), answer TRUE or FALSE and briefly justify your answers.

\subproblem $n^6 - n^4 = \Theta(n^6 - 5n^3)$.

\subproblem $\sqrt{n} = \Omega(n^{\sin{n}})$.

\subproblem $n! = O(2^n)$.

\subproblem $n^{100001}=o(1.00001^n)$. (Hint: L'H\^opital's rule)

\subproblem $3^n=5^{O(n)}$.

\subproblem Suppose that $f(n) = g(O(n))$. Give a counterexample to prove that this does not necessarily imply that $f(n) = O(g(n))$, and show why your counterexample works.


\begin{solution}
\subsolution True; both are polynomial and have highest power 6, so their asymptotic complexities are exactly the same.
\subsolution False; the given expression would mean that $n^{\sin n} \leq c\sqrt{n}$; for large enough values of $n$ where $\sin n = 1$, this becomes obviously false.
\subsolution False; the left side grows by increasing factors, while the right only grows by a factor of 2 for each increment in $n$.
\subsolution True; the given statement is equivalent to saying $\underset{n\rightarrow \infty}{lim} \frac{n^{100001}}{1.00001^n} = 0$; L'H\^opital's rule applied exactly 100002 times gives $\underset{n \rightarrow \infty}{lim} \frac{0}{1.00001^n * \ln(1.00001)^{100002}} = 0$.
\subsolution True; $3^n \leq 5^{\log_53 * n}$.
\subsolution Consider $f(n) = 3n^{3n}$; this is equal to $g(O(n))$ where $g(n) = n^n$ and $c = 3$ in $O(n) = cn$, but $O(g(n)) = O(n^n)$; the equation $3n^{3n} = O(n^n)$ is obviously not true since $n^{3n}$ grows much faster than $n^n$.
\end{solution}


\problem{6} {1/3 page}
Two languages $H$ and $K$ are \emph{recursively separable} if there
exists some recursive language $R$ such that $H \subseteq R$ and
$K \subseteq \overline{R}$.  Prove that the languages $L_1 =
\{\langle M \rangle : M  \mbox{ accepts } \langle M \rangle\}$ and
$L_2 = \{\langle M \rangle: M \mbox{
rejects } \langle M \rangle \}$ are not recursively separable.

\begin{solution}
Suppose that there is a decidable language $R$ such that $L_1 \subset \overline{R}$ and $L_2 \subset R$; this means that $R \cap L_1 = \emptyset$ (i.e. there are no machines in $R$ that accept their own descriptions).  $R$ has a Turing machine that accepts exactly $R$; let's call it $T$.  It's obvious that $\langle T \rangle$ is either in $L_1$ or $L_2$; suppose it's in $L_1$.  Then since $\langle T \rangle$ is in $L_1$, $T$ accepts $\langle T \rangle$.  But this means that $\langle T \rangle$ is in $R$, since $T$ accepts $\langle T \rangle$, which is a contradiction since the intersection of $R$ and $L_1$ is the empty set.  Hence, $\langle T \rangle$ cannot be in $L_1$.  Now suppose it's in $L_2$; that is, $T$ does not accept $\langle T \rangle$.  But since $L_2 \subset R$ and $\langle T \rangle \in L_2$, $\langle T \rangle \in R$; i.e. $T$ does accept $\langle T \rangle$, a contradiction.  Hence, there exists no such $R$.
\end{solution}

\problem{6}{1/3 page}
A function $f:\Sigma^* \rightarrow \Sigma^*$ is {\it computable} if
there exists a Turing Machine $M$ that, when given $s \in \Sigma^*$ as input,
halts with $f(s)$ written on the tape. The \emph{range} of $f$ is $\{f(x):
  x \in \Sigma^*\}$. Prove that a nonempty language is r.e.~if and only if 
  it is the range of a computable function. 

\begin{solution}
Suppose we have a nonempty recognizable language $L$.  Since $L$ is recognizable, there is an enumerator that prints out the strings in $L$ one at a time.  Moreover, we can construct a bijective function $g: \mathbb{N} \rightarrow \Sigma^*$ since $\Sigma^*$ is countable.  We construct $f$ using the enumerator that prints out the strings in $L$ as follows: for $s \in \Sigma^*$, $s = g(k)$ for some $k$.  Run the enumerator that prints out strings in $L$; $f(s)$ is equal to the $k$th string printed by the enumerator.  Thus, the output of $f$ is always some string in $L$, and we have constructed a function that prints out all strings in $L$.  We can construct a TM that halts with $f(s)$ on the tape as follows: given a string $s \in \Sigma^*$, calculate $k$ such that $g(k) = s$ (this is possible because $g$ is bijective and so can be inverted).  Then, run the enumerator for $L$ and count the strings that are printed.  When $s$ is printed out, take the current count and pass it into $g$, then erase the input tape and write the output of $g$ on the tape.  This process is guaranteed to halt since any particular string in $L$ must be printed out at some point.

Now suppose we have a computable function $f$; we want to show that the range of $f$ is recognizable.  To do this, we need to construct a TM that accepts a string iff it is in the range of $f$.  We do this using $g$ above; we define the following TM: 
\begin{itemize}
	\setlength\itemsep{0cm}
	\item Do the following starting from 0 and counting upwards:
	\begin{itemize}
		\item Compute $f(g(i))$, where $i$ is the current index.  If the result is equal to the input string, halt and accept.  Otherwise, continue to the next index.
	\end{itemize}
\end{itemize}

It should be clear that this process halts and accepts iff the input string is an output of the function $f$, and otherwise runs forever (i.e. rejects the input string).
\end{solution}

\problem{CHALLENGE! (3 extra credit points)}{1/2 page}
In this problem you will define a function that grows faster than any computable function. Thus you will prove the existence of an uncomputable function directly, without relying on Turing's diagonalization argument. 

The {\bf busy-beaver function}
$\beta(n)$ is
the largest number of $a$'s that can be printed by any $n$-state, two-symbol Turing machine that eventually halts when started from the empty tape.

Show that $\beta$ is not computable. 

\begin{solution}
  % Write your answer here.
\end{solution}

%%%%%%%%%%%%%%%%%%%%%%%%%%%%%%%%%%%%%%%%%%%%%%%
\PART{Madhu and Erin}
%%%%%%%%%%%%%%%%%%%%%%%%%%%%%%%%%%%%%%%%%%%%%%%

\problem{6+6}{1/2+2/3 pages}
Prove that the class ${\cal P}$ is closed under:
\subproblem Concatenation.
\subproblem Kleene star. (\emph{Hint:} Look at the algorithm we gave in class for recognizing context-free languages via Chomsky Normal Form.)

\begin{solution}
\subsolution Suppose we have languages $A$ and $B$ in ${\cal P}$.  To test if a string $s$ is in $A \circ B$, we only need to check every possible pairing of strings that could make $s$; to be exact, we just split $s$ into two strings at every possible place and check if the first string is in $A$ and the second string is in $B$.  Since $A$ and $B$ are decidable languages in ${\cal P}$, they have Turing machines with polynomial runtimes; let's call them $T_1$ and $T_2$, with corresponding polynomial runtimes $O(n^p)$ and $O(n^q)$.  We construct the following Turing machine for some input string $s$:
\begin{itemize}
	\setlength\itemsep{0cm}
	\item Do the following for all possible pairs of strings that can be concatenated to make $s$:
	\begin{itemize}
		\item Pass the first string in the pair to $T_1$.  If $T_1$ accepts, move on; if it rejects, reject.
		\item Pass the second string in the pair to $T_2$.  If $T_2$ accepts, accept $s$; if it rejects, reject.
	\end{itemize}
\end{itemize}

This only accepts if $s \in A \circ B$, and has polynomial run-time $O(n(O(n^p)+O(n^q))) = O(n^{\textsc{max}(p,q)+1})$ since the loop is run at most $n$ times and each runthrough takes $O(n^p)+O(n^q)$ runtime.

\subsolution Suppose we have a language $L \in {\cal P}$, and a corresponding TM $T$ with runtime $p(n)$.  We construct the following TM $T'$: 
\begin{itemize}
	\setlength\itemsep{0cm}
	\item For an input string $s$ of length $n$, do the following for each of $i = 1$ through $i = n$:
	\begin{itemize}
		\item For each of $j = 1$ through $j = n - i + 1$, do the following:
		\begin{itemize}
			\item Pass the substring of $s$ starting at index $j$ of length $i$ to $T$.  $T$ must either accept or reject because its language is in ${\cal P}$.  If $T$ accepts, keep this substring in memory.  If $T$ rejects, do the following for $k = j$ through $k = j + i - 1$:
			\begin{itemize}
				\item If the string starting at $j$ and ending at index $k$ AND the string starting at index $k+1$ and ending at index $i + j - 1$ are both saved (the empty string counts as saved), save this substring as well.
			\end{itemize}
		\end{itemize}
	\end{itemize}
	\item At the end of the above loops, if the entire string has been saved, accept.  Otherwise, reject.
\end{itemize}

The above process works by checking every possible substring of the input string and building upwards from there to check if the total string is a composite of accepted strings.  For example, if the input string is $abcd$, it would check $a$, $b$, $c$, and $d$ in turn, then check $ab$, $bc$, $cd$, then check $abc$ and $bcd$, and finally $abcd$.  The outermost loop of this algorithm is run $n$ times; the next loop is run approximately $n$ times per outer loop, and the inner loop takes $p(n) + O(n)$ at most.  Hence, the runtime of this algorithm is $O(n) * O(n) * (p(n) + O(n))$, or $O(n^{2+\textsc{max}(k, 1)})$, where $p(n) = O(n^k)$.

\end{solution}

\problem{8}{.5 pages}
Prove that $\textsc{ALL}_\textsc{DFA} = \{\langle D \rangle : D$ is a DFA and $L(D)=\Sigma^*\}$ is in $\mathcal{P}$.

\begin{solution}
This is simply a matter of checking all possible paths in a DFA $D$ to make sure that they end in an accept state.  In particular, every state along any path must be an accept state, or the string that led to that path is not accepted by the DFA.  We write a TM $T$ that decides $\textsc{ALL}_\textsc{DFA}$ as follows:

\begin{itemize}
	\setlength\itemsep{0cm}
	\item Check that an input string $s$ encodes a valid DFA $D$; if no, reject.  Otherwise, decode into a DFA $D = (Q, \Sigma, q_0, \delta, F)$.  
	\item Mark the start state of $D$ (which is $q_0$).
	\item Mark any states that can be transitioned to from any already-marked states.  Repeat this step while this step successfully marks a state.
	\item If any states encountered in steps 2 and 3 are not accept states, reject.  Otherwise, continue.
	\item If step 3 terminates without rejecting, accept because all reachable states in $D$ are accept states and therefore $D$ accepts all strings.
\end{itemize}

This machine clearly halts, and decides whether an input DFA $D$ accepts all strings or not.  The polynomial runtime of this machine depends on both the length of the input string and the number of states in the DFA; step 1 takes $O(n)$, where $n$ is the length of the input string, and steps 2 and 3 take at MOST $O(q)$ time, where $q$ is the number of states in $D$.  Thus, the runtime of this algorithm is $O(\textsc{max}(n,q))$.
\end{solution}
\end{document}
